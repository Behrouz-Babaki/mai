\documentclass{ximera}

\title{Probability}

\begin{document}
% \begin{abstract}
% The probability portion of the self-evaluation test for the University
% of Leuven's Masters of Artificial Intelligence program.
% \end{abstract}
\maketitle


% Question 10
\begin{question}
Box A contains balls with numbers from 1 to 6.  Box B contains balls
numbered 1 to 4.  I draw a random ball from A and one from B.  What is
the probability the sum of both numbers is 7?
\begin{solution}
The answer is \answer{1/6}.
\end{solution}
Let's show each outcome with an ordered pair (in which the first
element shows the number of first ball and the second element shows
the number of the second ball). All possible outcomes are listed below
(The outcomes with a sum of 7 are specified with boldface):


\[
\begin{matrix}
(1,1) & (2,1) & (3,1) & (4,1) & (5,1) & \textbf{(6,1)} \\
(1,2) & (2,2) & (3,2) & (4,2) & \textbf{(5,2)} & (6,2) \\
(1,3) & (2,3) & (3,3) & \textbf{(4,3)} & (5,3) & (6,3) \\
(1,4) & (2,4) & \textbf{(3,4)} & (4,4) & (5,4) & (6,4)
\end{matrix}
\]
\[
Pr (\text{sum = 7}) = \frac{N(\text{sum = 7})}{N(\text{total})} =
\frac{4}{24} = \frac{1}{6}
\]
\end{question}

% Question 11
\begin{question}
Box A contains 5 white and 2 black balls.  Box B contains 3 white and
3 black balls.  Someone chose a box randomly (1/2 chance each) and
took a random ball from it.  It is black.  What is the probability
that the ball was taken from box B?
\begin{solution}
The answer is \answer{7/11}.
\end{solution}
Using Bayes rule, we have:
\begin{align*}
Pr (\text{box B} | \text{black}) &= \frac{Pr(\text{black} | \text{box B}) Pr(\text{box B})}{Pr(\text{black} | \text{box B}) Pr(\text{box B}) + Pr(\text{black} | \text{box A}) Pr(\text{box A})} \\
&=\frac{(\frac{3}{6}) \times (\frac{1}{2})}{(\frac{3}{6}) \times (\frac{1}{2}) + (\frac{2}{7}) \times (\frac{1}{2})} = \frac{7}{11}
\end{align*}
\end{question}

% Question 12
\begin{question}
There are two different train connections between cities A and B.  One
goes via C, the other via D.  30\% of all trains go via C; among
these, 10\% are delayed upon arrival.  70\% go via D; among these, 5\%
are delayed.  A train from A arriving in B is delayed.  What is the
probability it went via C?
\begin{solution}
The answer is \answer{6/13}.
\end{solution}
Using Bayes rule,


\begin{align*}
Pr (\text{via C} | \text{delayed}) &= \frac{Pr(\text{delayed} | \text{via C}) Pr(\text{via C})}{Pr(\text{delayed} | \text{via C}) Pr(\text{via C}) + Pr(\text{delayed} | \text{via D}) Pr(\text{via D})} \\
& = \frac{(0.1)(0.3)}{(0.1)(0.3) + (0.05)(0.7)} = \frac{6}{13}
\end{align*}
\end{question}

% Question 13
\begin{question}
Someone has rolled two dice.  Given that the total is even, what is
the probability that it is less than 9?
\begin{solution}
The probability of this event is equal to \answer{7/9}.
\end{solution}
Rolling two dice has 36 possible outcomes. Outcomes with a total smaller than 9 are listed below. Among these outcomes, those with an even sum are specified by boldface:


\[
\begin{matrix}
\mathbf{(1,1)} & (1,2) & \mathbf{(1,3)} & (1,4) & \mathbf{(1,5)} & (1,6) \\
(2,1) & \mathbf{(2,2)} & (2,3) & \mathbf{(2,4)} & (2,5) & \mathbf{(2,6)} \\
\mathbf{(3,1)} & (3,2) & \mathbf{(3,3)} & (3,4) & \mathbf{(3,5)} && \\
(4,1) & \mathbf{(4,2)} & (4,3) & \mathbf{(4,4)} &&& \\
\mathbf{(5,1)} & (5,2) & \mathbf{(5,3)} &&&& \\
(6,1) & \mathbf{(6,2)}
\end{matrix}
\]
Using Bayes rule,
\[
Pr (\text{less than 9} | \text{even}) = \frac{Pr(\text{less than 9},
  \text{even})}{Pr(\text{even})} =
\frac{(\frac{14}{36})}{(\frac{18}{36})} = \frac{7}{9}
\]
\end{question}

% Question 14 %
\begin{question}
A lottery ticket costs 5 dollars.  There is no limit to the number of
tickets that can be sold, and each sold ticket has a probability of
2\% of winning 100 dollars.  You buy three tickets, what is your expected
payout?

\begin{solution}
The answer is \answer{-9}.
\end{solution}

The expected payout of three tickets is equal to the sum of expected
payout of each of three tickets.
\[
E[X_1 + X_2 + X_3] = E[X_1] + E[X_2] + E[X_3] = 3 \times E[X]
\]

\[
E[X] = -5 + \left(Pr(X = 100) \times (100) + Pr(X = 0) \times (0)\right) = -5 + 0.02 \times 100 = -3
\]
Hence the expected payout of three tickets would be $3 \times -3 = -9$.
\end{question}

\end{document}
