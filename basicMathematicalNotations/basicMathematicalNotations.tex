\documentclass{ximera}

\title{Basic Mathematical Notations}

\begin{document}
% \begin{abstract}
% The mathematics portion of the self-evaluation test for the University
% of Leuven's Masters of Artificial Intelligence program.
% \end{abstract}
\maketitle

% Question 1
\begin{question}
What is the result of $\sum_{i=2}^6 i$?
\begin{solution}
The sum is equal to \answer{20}.
\end{solution}
This sum could be computed via
\[
\sum_{i=2}^6 i = 2 + 3 + 4 + 5 + 6,
\]
or by using the formula
\[
\sum_{i=1}^n i = \frac{n(n+1)}{2}.
\]
\end{question}

% Question 2
\begin{question}
What is  the result of $\prod_{i=2}^{5} i$ ? 
\begin{solution}
The product is equal to \answer{120}.
\end{solution}
$\prod_{i=2}^5 i = 2 \times 3 \times 4 \times 5 = 120$
\end{question}

% Question 3
\begin{question}
What is  the result of  $\prod_{i=1}^5 2^i$  ? 
\begin{solution}
The product is equal to \answer{32768}.
\end{solution}
\begin{align*}
\prod_{i=1}^5 2^i &= 2^1 \times 2^2 \times 2^3 \times 2^4 \times 2^5 \\
&= 2^{1 + 2 + 3 + 4 + 5} \\
&= 2^{15} \\
&= 32768. 
\end{align*}
\end{question}

% Question 5
\begin{question}
What is  the result of  $\log_2 2^k 4^m$  ? 
\begin{solution}
The answer is \answer{$k+2m$}.
\end{solution}
\begin{align*}
\log_2{2^k 4^m} &= \log_2{2^k} + \log_2{4^m} \\
&= k \log_2{2} + m \log_2{4} \\
&= k + 2m.
\end{align*}
\end{question}


\end{document}
