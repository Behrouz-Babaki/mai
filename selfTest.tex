\documentclass{ximera}

\title{Self test}

\begin{document}
\begin{abstract}
A self-evaluation for the University of Leuven's Masters of Artificial
Intelligence program.
\end{abstract}
\maketitle

\subsection*{Mathematics}

\begin{question}
What is the result of $\sum_{i=2}^6 i$?
\begin{solution}
The sum is equal to \answer{20}.
\end{solution}
This sum could be computed via
\[
\sum_{i=2}^6 i = 2 + 3 + 4 + 5 + 6,
\]
or by using the formula
\[
\sum_{i=1}^n i = \frac{n(n+1)}{2}.
\]
\end{question}

\subsection*{Set theory}

\begin{question}
Let $A$ be the set of all animals, $B$ the set of all birds, and $F$
the set of all flying animals. Which of the following statements are correct?
\begin{solution}
\begin{multiple-choice}
\choice[correct]{$B\cap F$ is the set of all birds that can fly.}
\choice[correct]{$B\setminus F$ is the set of all birds that do not fly.}
\choice[correct]{$F\subseteq A$.}
\choice{$F\subseteq B$.}
\choice[correct]{$B\subseteq A$.}
\end{multiple-choice}
\end{solution}
\end{question}


\end{document}
