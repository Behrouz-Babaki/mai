\documentclass{ximera}

\usepackage{algorithm}
\usepackage{algorithmic}

\title{Self test}

\begin{document}
\begin{abstract}
A self-evaluation for the University of Leuven's Masters of Artificial
Intelligence program.
\end{abstract}
\maketitle

\subsection*{Mathematics}

% Question 1
\begin{question}
What is the result of $\sum_{i=2}^6 i$?
\begin{solution}
The sum is equal to \answer{20}.
\end{solution}
This sum could be computed via
\[
\sum_{i=2}^6 i = 2 + 3 + 4 + 5 + 6,
\]
or by using the formula
\[
\sum_{i=1}^n i = \frac{n(n+1)}{2}.
\]
\end{question}

% Question 2
\begin{question}
What is  the result of $\prod_{i=2}^{5} i$ ? 
\begin{solution}
The product is equal to \answer{120}.
\end{solution}
$\prod_{i=2}^5 i = 2 \times 3 \times 4 \times 5 = 120$
\end{question}

% Question 3
\begin{question}
What is  the result of  $\prod_{i=1}^5 2^i$  ? 
\begin{solution}
The product is equal to \answer{32768}.
\end{solution}
\begin{align*}
\prod_{i=1}^5 2^i &= 2^1 \times 2^2 \times 2^3 \times 2^4 \times 2^5 \\
&= 2^{1 + 2 + 3 + 4 + 5} \\
&= 2^{15} \\
&= 32768. 
\end{align*}
\end{question}

% Question 5
\begin{question}
What is  the result of  $\log_2 2^k 4^m$  ? 
\begin{solution}
The answer is \answer{$k+2m$}.
\end{solution}
\begin{align*}
\log_2{2^k 4^m} &= \log_2{2^k} + \log_2{4^m} \\
&= k \log_2{2} + m \log_2{4} \\
&= k + 2m.
\end{align*}
\end{question}

\subsection*{Combinatorics}

% Question 6
\begin{question}
In a group of 15 people, everyone shakes hands with everyone else.  How many handshakes are there? 
\begin{solution}
The answer is \answer{105}.
\end{solution}
Every handshake requires two individuals. So the total number of handshakes is equal to the number of ways that we could pick 2 individuals from a group of 15:
\[
\binom{15}{2} = \frac{15!}{(15 - 2)! \cdot 2!} = 105
\]
\end{question}

% Question 7
\begin{question}
A Web search query returns ten answers. How many possible ways are there to rank these ten answers?
\begin{solution}
There are \answer{3628800} ways to rank the answers.
\end{solution}
The possible ways to rank these ten answers are equal to the number of permutations of 10 objects, which is $10! = 3628800$.
\end{question}

% Question 8
\begin{question}
How many subsets are there of $\{1, ... , 100\}$ that do not contain an even number ? 
\begin{solution}
There are \answer{$2^{50}$} subsets.
\end{solution}
Each subset of $\{1 , \ldots , 100\}$ which does not contain an even number is a subset of $\{1, 3, 5, \ldots , 97, 99\}$. So it would suffice to count the number of subsets of $\{1, 3, 5, \ldots , 97, 99\}$. We know that a set containing $n$ elements has $2^n$ subsets (including the empty subset). So the answer is $2^{50}$.
\end{question}

% Question 9
\begin{question}
10 toys are in a box.  A child can choose 3 of them.  How many different choices can it make (disregarding the order in which it chooses the toys)?  
\begin{solution}
There are \answer{120} choices.
\end{solution}
The order is not important, hence the answer is
\[
\binom{10}{3} = \frac{10!}{(10-3)! \cdot 3!} = 120
\]
\end{question}

\subsection*{Probability}

% Question 10
\begin{question}
Box A contains balls with numbers from 1 to 6.  Box B contains balls numbered 1 to 4.   I draw a random ball from A and one from B.  What is the probability the sum of both numbers is 7?
\begin{solution}
The answer is \answer{$\frac{1}{6}$}.
\end{solution}
Let's show each outcome with an ordered pair (in which the first element shows the number of first ball and the second element shows the number of the second ball). All possible outcomes are listed below (The outcomes with a sum of 7 are specified with boldface):
\[
\begin{matrix}
(1,1) & (2,1) & (3,1) & (4,1) & (5,1) & \textbf{(6,1)} \\
(1,2) & (2,2) & (3,2) & (4,2) & \textbf{(5,2)} & (6,2) \\
(1,3) & (2,3) & (3,3) & \textbf{(4,3)} & (5,3) & (6,3) \\
(1,4) & (2,4) & \textbf{(3,4)} & (4,4) & (5,4) & (6,4)
\end{matrix}
\]
\[
Pr (\text{sum = 7}) = \frac{N(\text{sum = 7})}{N(\text{total})} = \frac{4}{24} = \frac{1}{6}
\]
\end{question}

% Question 11
\begin{question}
Box A contains 5 white and 2 black balls.   Box B contains 3 white and 3 black balls.  Someone chose a box randomly (1/2 chance each) and took a random ball from it.  It is black.  What is the probability that the ball was taken from box B?
\begin{solution}
The answer is \answer{$\frac{7}{11}$}.
\end{solution}
Using Bayes rule, we have:
\begin{align*}
Pr (\text{box B} | \text{black}) &= \frac{Pr(\text{black} | \text{box B}) Pr(\text{box B})}{Pr(\text{black} | \text{box B}) Pr(\text{box B}) + Pr(\text{black} | \text{box A}) Pr(\text{box A})} \\
&=\frac{(\frac{3}{6}) \times (\frac{1}{2})}{(\frac{3}{6}) \times (\frac{1}{2}) + (\frac{2}{7}) \times (\frac{1}{2})} = \frac{7}{11}
\end{align*}
\end{question}

% Question 12
\begin{question}
There are two different train connections between cities A and B.  One goes via C, the other via D.  30\% of all trains go via C; among these, 10\% are delayed upon arrival.  70\% go via D; among these, 5\% are delayed.   A train from A arriving in B is delayed.  What is the probability it went via C?
\begin{solution}
The answer is \answer{$\frac{6}{13}$}.
\end{solution}
Using Bayes rule,
\begin{align*}
Pr (\text{via C} | \text{delayed}) &= \frac{Pr(\text{delayed} | \text{via C}) Pr(\text{via C})}{Pr(\text{delayed} | \text{via C}) Pr(\text{via C}) + Pr(\text{delayed} | \text{via D}) Pr(\text{via D})} \\
& = \frac{(0.1)(0.3)}{(0.1)(0.3) + (0.05)(0.7)} = \frac{6}{13}
\end{align*}
\end{question}

% Question 13
\begin{question}
Someone has rolled two dice.  Given that the total is even, what is the probability that it is less than 9?
\begin{solution}
The probability of this even is equal to \answer{$\frac{7}{9}$}.
\end{solution}
Rolling two dice has 36 possible outcomes. Outcomes with a total smaller than 9 are listed below. Among these outcomes, those with an even sum are specified by boldface:
\begin{equation*}
\begin{matrix}
\textbf{(1,1)} & (1,2) & \textbf{(1,3)} & (1,4) & \textbf{(1,5)} & (1,6) \\
(2,1) & \textbf{(2,2)} & (2,3) & \textbf{(2,4)} & (2,5) & \textbf{(2,6)} \\
\textbf{(3,1)} & (3,2) & \textbf{(3,3)} & (3,4) & \textbf{(3,5)} && \\
(4,1) & \textbf{(4,2)} & (4,3) & \textbf{(4,4)} &&& \\
\textbf{(5,1)} & (5,2) & \textbf{(5,3)} &&&& \\
(6,1) & \textbf{(6,2)}
\end{matrix}
\end{equation*}
Using Bayes rule,
\begin{equation*}
Pr (\text{less than 9} | \text{even}) = \frac{Pr(\text{less than 9}, \text{even})}{Pr(\text{even})} = \frac{(\frac{14}{36})}{(\frac{18}{36})} = \frac{7}{9}
\end{equation*}
\end{question}

% Question 14 %
\begin{question}
A lottery ticket costs \$5. 
There is no limit to the number of tickets that can be sold, and each sold ticket has a probability of 2\% of winning \$100.
You buy three tickets, what is your expected payout?
The expected payout of three tickets is equal to the sum of expected payout of each of three tickets. 
\begin{solution}
The answer is \answer{-9}.
\end{solution}
\begin{equation*}
E[X_1 + X_2 + X_3] = E[X_1] + E[X_2] + E[X_3] = 3 \times E[X]
\end{equation*}

\begin{equation*}
E[X] = -5 + \left(Pr(X = 100) \times (100) + Pr(X = 0) \times (0)\right) = -5 + 0.02 \times 100 = -3
\end{equation*}
Hence the expected payout of three tickets would be $3 \times -3 = -9$.
\end{question}


\subsection*{Algebra}

% Question 17
\begin{question}
Solve for $x_2$ in the following system of equations:
\begin{eqnarray*}
-2x_1 + 4x_2 & = &-6\\
3x_1  - 9x_2 & = & 12 
\end{eqnarray*}
\vspace*{13pt}
\begin{solution}
The value of $x_1$: \answer{1}.
The value of $x_2$: \answer{-1}.
\end{solution}
Multiplying the first equation by 3 and the second one by 2, we'll get the following equivalent system:
\begin{alignat*}{2}
 -6  x_1  + 12  x_2 & = -18 \\
 6  x_1  - 18  x_2 & = 24
\end{alignat*}
By summing these two equations we'll get $x_2 = -1$, and by replacing this value for $x_2$ in any of the equations, we'll obtain $x_1 = 1$.
\end{question}

% Question 18
\begin{question}
Multiply the following matrices together:
%\begin{equation}
\[ \left( \begin{array}{cc}
1 & 2 \\
3 & 5 \\
2 & 8
\end{array} \right)
%
\left( \begin{array}{ccc}
2 & 6 & 1 \\
3 & 3 & 3
\end{array} \right)
\]
\begin{solution}
% TODO (Behrouz) add matrix-format solution here
\end{solution}
This is the formula for multiplication of two matrices $A_{m \times n}$ and $B_{n \times p}$:
\[
    (AB)_{ij} = \sum_{k=1}^n A_{ik}B_{kj}. 
\]
Using this formula, we could compute the multiplication of the two given matrices:
\[
\begin{pmatrix}
8 & 12 & 7 \\
21 & 33 & 18 \\
28 & 36 & 26
\end{pmatrix}
\]
\end{question}

\subsection*{Algorithms}

% Question 19
% TODO (Behrouz) Verify if this is the right way of putting multiple-section questions
\begin{question}
Consider the algorithm below.
\begin{algorithm}
\caption{Function(Array $A$)}
\begin{algorithmic}
  \FOR{$i = 1$ to $A$.length}
  \FOR{$j = 1$ to $A$.length--i}
  \IF{$A[i+1] < A[i]$}
  \STATE $temp = A[i]$
  \STATE $A[i] = A[i+1]$
  \STATE $A[i+1] = temp$
  \ENDIF
  \ENDFOR
  \ENDFOR
  \STATE output $A$
\end{algorithmic}
\end{algorithm}

% part 1 of question
If the algorithm is invoked on the array $[ 5~~ 7~~ 3~~ 9]$ what is the output of the algorithm?
\begin{solution}
% TODO (Behrouz) Change this into a row vector with constant #columns
The output is \answer{$ [3 \quad 5 \quad 7 \quad 9] $}.
\end{solution}

% part 2 of question
What does the algorithm do? 
% TODO (Behrouz) Verify if this is a correct implementation of free-response
\begin{free-response}
\end{free-response}
This algorithm sorts the input array using a technique called bubble sort.

% part 3 of question
Given an array with ten elements, how many comparisons (that is, $<$) operations does the following piece of code perform?
\begin{solution}
There are \answer{$45$} comparisons.
\end{solution}  
The number of iterations of the inner loop starts with value 9 and gradually decreases to 1. So the total number of comparisons is $ 9 + 8 + \ldots + 2 + 1 = 45$.

% part 4 of question
Can you generalize your answer such that it gives an upperbound on the number of comparison operations for an array of length $n$?
\begin{solution}
The smallest upper bound (expressed as a function of $n$) is \answer{$n^2$}.
\end{solution}
\begin{hint}
Are you familiar with asymptotic analysis of algorithms (especially the Big-$\Theta$ notation)? 
\end{hint}
For an array of length $n$, the number of iterations of inner loop starts with value $n-1$ and gradually decreases to 1. So the total number of iterations would be $(n-1) + (n-2) + \ldots + 1 = \frac{n(n-1)}{2}$. This indicates that the running time of this algorithm is of $\Theta(n^2)$.
\end{question}

% Question 21
\begin{question}
What is the maximal number of leaves a sorted binary tree with depth $4$ can possess?
\begin{solution}
The answer is \answer{$16$}.
\end{solution}
The maximal number of nodes at depth-level $d$ of a binary tree is $2^d$. So the answer to this question is $2^4 = 16$. 
\end{question}

% Question 22
\begin{question}
Order the following functions according to their growth-rate (use the theory of big-Oh notation). Put the slower growing functions first.

$ n^7, 8n!, 5 \log_2 n,~ 3n, 2^n, ~ 7.5 n\log_2 n, ~ 6 n^2$ 

% TODO (Behrouz) Present the solution as a row vector 
\begin{solution}
\end{solution}
\begin{equation*}
5 \log_2^n \prec 3n \prec 7.5 n \log_2{n} \prec 6n^2 \prec n^7 \prec 2^n \prec 8n!
\end{equation*}
\end{question}

\subsection*{Boolean Logic}

% Question 23
\begin{question}
Which of the following statements are correct?
\begin{solution}
\begin{multiple-choice}
\choice[correct]{not(A and B) is true whenever A is false or B is false}
\choice[correct]{not (A or B) is true if and only if (not A and not B) is true}
\choice{not (A or B) is true if and only if  (not A or not B) is true}
\choice[correct]{not (A and B) is true whenever A is false}
\end{multiple-choice}
\end{solution}
\end{question}

% Question 24
\begin{question}
Simplify the following boolean expression: \\
(A and B) or (A and B and C) or (A and B and C) or (A and B and C and D) or (A and B and C and D).
\begin{solution}
% TODO (Behrouz) There should be a better way of doing this
The answer is \answer{A and B}. 
\end{solution}
\end{question}


\subsection*{Set theory}
\begin{question}
Box A contains balls with numbers from 1 to 6.  Box B contains balls numbered 1 to 4.   I draw a random ball from A and one from B.  What is the probability the sum of both numbers is 7?
\end{question}

\begin{question}
Let $A$ be the set of all animals, $B$ the set of all birds, and $F$
the set of all flying animals. Which of the following statements are correct?
\begin{solution}
\begin{multiple-choice}
\choice[correct]{$B\cap F$ is the set of all birds that can fly.}
\choice[correct]{$B\setminus F$ is the set of all birds that do not fly.}
\choice[correct]{$F\subseteq A$.}
\choice{$F\subseteq B$.}
\choice[correct]{$B\subseteq A$.}
\end{multiple-choice}
\end{solution}
\end{question}


\end{document}
