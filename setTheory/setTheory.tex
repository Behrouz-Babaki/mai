\documentclass{ximera}

\title{Set Theory}

\begin{document}
\begin{abstract}
The set theory portion of the self-evaluation test for the
University of Leuven's Masters of Artificial Intelligence program.
\end{abstract}
\maketitle

% Question 26
Pick the correct answer:
\begin{question}
$(C \setminus A) \setminus B = C \setminus (A \cup B)$ :
\begin{solution}
\begin{multiple-choice}
\choice[correct]{always true}
\choice{always false}
\choice{depends on sets}
\end{multiple-choice}
\end{solution}
\end{question}

\begin{question}
$C \setminus (A \cup B) = (C \setminus A) \cap (C\setminus B)$ :
\begin{solution}
\begin{multiple-choice}
\choice[correct]{always true}
\choice{always false}
\choice{depends on sets}
\end{multiple-choice}
\end{solution}
\end{question}


% Question 27
Let $A$ be the set of all animals, $B$ the set of all birds, and $F$
the set of all flying animals. Are the following statements true or false?

\begin{question}
$B\cap F$ is the set of all birds that can fly.
\begin{solution}
\begin{multiple-choice}
\choice[correct]{true}
\choice{false}
\end{multiple-choice}
\end{solution}
\end{question}

\begin{question}
$B\setminus F$ is the set of all birds that do not fly.
\begin{solution}
\begin{multiple-choice}
\choice[correct]{true}
\choice{false}
\end{multiple-choice}
\end{solution}
\end{question}

\begin{question}
$F\subseteq B$.
\begin{solution}
\begin{multiple-choice}
\choice{true}
\choice[correct]{false}
\end{multiple-choice}
\end{solution}
\end{question}

% Question 28
\begin{question}
Simplify $(A \cap B) \cup (A \cap B^C)$ where $B^C$ is the complement of $B$.
\begin{solution}
The answer is \answer{$A$}.
\end{solution}
$(A \cap B) \cup (A \cap B^C) = A \cap (B \cup B^C) = A \cap \mathbb{U} = A$
\end{question}

% Question 29
\begin{question}
A survey of 500 students taking one or more courses in algebra,
physics and statistics during one semester revealed the following
number of students in the indicated subjects : 329 in Algebra, 186 in
Physics, 295 in Statistics, 83 taking Algebra and Physics, 217 Algebra
and Statistics, and 63 Physics and Statistics.  How many students took
all three subjects ?
\begin{solution}
The answer is \answer{53}. 
\end{solution}

Let us denote the set of students taking courses Algebra, Physics, and
Statistics by letters $A$, $P$, and $S$, respectively. We have this
information about carnality of these three sets:

\begin{equation*}
	\begin{array}{lll}
	 |A| = 329							& |P| = 186				& |S| = 295 \\
	 |A \cap P| = 83				& |A \cap S| = 217	& |P \cap S| = 63 \\
	 |A \cup P \cup S| = 500	& &
	\end{array}
\end{equation*}

According to the principle of inclusion-exclusion, for three sets A, P and S we have:
\begin{align*}
|A \cup P \cup S| &= |A| + |P| + |S| - |A \cap P| - |A \cap S| - |P \cap S| + |A \cap P \cap S|\\
|A \cap P \cap S| &= 500 - 329 - 186 - 295 + 83 + 217 + 63 = 53
\end{align*}
\end{question}



\end{document}
