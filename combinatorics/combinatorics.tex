\documentclass{ximera}

\title{Combinatorics}

\begin{document}
\begin{abstract}
The combinatorics portion of the self-evaluation test for the
University of Leuven's Masters of Artificial Intelligence program.
\end{abstract}
\maketitle



% Question 6
\begin{question}
In a group of 15 people, everyone shakes hands with everyone else.
How many handshakes are there?
\begin{solution}
The answer is \answer{105}.
\end{solution}
Every handshake requires two individuals. So the total number of
handshakes is equal to the number of ways that we could pick 2
individuals from a group of 15:
\[
\binom{15}{2} = \frac{15!}{(15 - 2)! \cdot 2!} = 105
\]
\end{question}

% Question 7
\begin{question}
A Web search query returns ten answers. How many possible ways are
there to rank these ten answers?
\begin{solution}
There are \answer{3628800} ways to rank the answers.
\end{solution}
The possible ways to rank these ten answers are equal to the number of
permutations of 10 objects, which is $10! = 3628800$.
\end{question}

% Question 8
\begin{question}
How many subsets are there of $\{1, ... , 100\}$ that do not contain
an even number ?
\begin{solution}
There are \answer{$2^{50}$} subsets.
\end{solution}
Each subset of $\{1 , \ldots , 100\}$ which does not contain an even
number is a subset of $\{1, 3, 5, \ldots , 97, 99\}$. So it would
suffice to count the number of subsets of $\{1, 3, 5, \ldots , 97,
99\}$. We know that a set containing $n$ elements has $2^n$ subsets
(including the empty subset). So the answer is $2^{50}$.
\end{question}

% Question 9
\begin{question}
10 toys are in a box.  A child can choose 3 of them.  How many
different choices can it make (disregarding the order in which it
chooses the toys)?
\begin{solution}
There are \answer{120} choices.
\end{solution}
The order is not important, hence the answer is
\[
\binom{10}{3} = \frac{10!}{(10-3)! \cdot 3!} = 120
\]
\end{question}


\end{document}
